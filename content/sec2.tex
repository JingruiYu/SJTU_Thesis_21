\newpage
\fancyhead[LH]{上海交通大学学位论文}
\fancyhead[RH]{第二章\quad正文文字格式}
\section{正文文字格式}
\subsection{论文正文}
论文正文是主体,一般由标题、文字叙述、图、表格和公式等部分构成。一般可包括理论分析、计算方法、实验装置和测试方法,经过整理加工的实验结果分析和讨论,与理论计算结果的比较以及本研究方法与已有研究方法的比较等,因学科性质不同可有所变化。\par
论文内容一般应由十个主要部分组成,依次为:⒈封面,⒉中文摘要,⒊英文摘要,⒋目录,⒌符号说明,⒍论文正文,⒎参考文献,⒏附录,⒐致谢,⒑攻读学位期间发表的学术论文目录。\par
以上各部分独立为一部分,每部分应从新的一页开始,且纸质论文应装订在论文的右侧。\par
\subsection{字数要求}
\subsubsection{硕士论文字数要求}
各学科和学部自定
\subsubsection{博士论文字数要求}
各学科和学部自定
\subsection{本章小结}
本章介绍了……

\newpage
\fancyhead[LH]{上海交通大学学位论文}
\fancyhead[RH]{第三章\quad图表、公式格式}
\section{图表、公式格式}
\subsection{图表格式}

\begin{figure}[htb] 
\center{\includegraphics[width=0.95\textwidth]  {fig2.png}} 
\caption{内热源沿径向的分布}
\end{figure}

\begin{table}[!htbp]
\centering
\caption{高频感应加热的基本参数}
\begin{tabular}{|c| c|c|c|}
\hline
感应频率 &感应发生器功率 & 工件移动速度  &感应圈与零件间隙\\
(KHz)&($\% \times$80Kw) &(mm/min)  &(mm)\\
\hline
250 &88 &5900 &1.65\\
\hline
250 &88 &5900 &1.65\\
\hline
250 &88 &5900 &1.65\\
\hline
250 &88 &5900 &1.65\\
\hline
250 &88 &5900 &1.65\\
\hline
250 &88 &5900 &1.65\\
\hline
250 &88 &5900 &1.65\\
\hline
250 &88 &5900 &1.65\\
\hline
\end{tabular}
\end{table}

\begin{table}
\centering
\captionsetup{singlelinecheck=off}
\caption*{续表} %取消编号
\begin{tabular}{|c| c|c|c|}
\hline
感应频率 &感应发生器功率 & 工件移动速度  &感应圈与零件间隙\\
(KHz)&($\% \times$80Kw) &(mm/min)  &(mm)\\
\hline
250 &88 &5900 &1.65\\
\hline
250 &88 &5900 &1.65\\
\hline
\end{tabular}
\end{table}
%表格太大需要转页时,需要在续表上方注明“续表”,表头也应重复排出。


\subsection{公式格式}

\vspace{-10mm}
\begin{eqnarray}
\frac{1}{\mu} \nabla^2A - j \omega \sigma A -\nabla(\frac{1}{\mu}) \times(\nabla \times A)+J_0=0
\end{eqnarray}

\subsection{本章小结}
本章介绍了……

\newpage
\fancyhead[LH]{上海交通大学学位论文}
\fancyhead[RH]{第四章\quad全文总结}
\section{全文总结}

\subsection{主要结论}
本文主要……

\subsection{研究展望}
更深入的研究……